\chapter{Contexto Geológico}

Para ilustrar a completa ades\~ao ao estilo de cita{\c c}\~oes e listagem de
refer\^encias bibliogr\'aficas, a Tabela~\ref{tab:citation} apresenta cita{\c
c}\~oes de alguns dos trabalhos, utilizando o estilo alfabético (default). Para
utilização do estilo numérico, deve-se utilizar a opção number da classe ON, ou seja,
basta usar \verb|\documentclass[dsc, numbers]{on}|.

\begin{table}[h]
\caption[Exemplos de tabela (texto do índice)]{Exemplos de tabela mostrando os comandos para
  cita{\c c}\~oes utilizando o comando padr\~ao \texttt{\textbackslash citep} do \LaTeX\ e
  o comando \texttt{\textbackslash citet},
  fornecido pelo pacote \texttt{natbib}.}
\label{tab:citation}
\centering
{\footnotesize
\begin{tabular}{|c|c|c|}
  \hline
  Tipo da Publica{\c c}\~ao & \verb|\citep| & \verb|\citet|\\
  \hline
  Livro & \citep{book-example} & \citet{book-example}\\
  Artigo & \citep{article-example} & \citet{article-example}\\
  Relat\'orio & \citep{techreport-example} & \citet{techreport-example}\\
  Relat\'orio & \citep{techreport-exampleIn} & \citet{techreport-exampleIn}\\
  Anais de Congresso & \citep{inproceedings-example} &
    \citet{inproceedings-example}\\
  S\'eries & \citep{incollection-example} & \citet{incollection-example}\\
  Em Livro & \citep{inbook-example} & \citet{inbook-example}\\
  Disserta{\c c}\~ao de mestrado & \citep{mastersthesis-example} &
    \citet{mastersthesis-example}\\
  Tese de doutorado & \citep{phdthesis-example} & \citet{phdthesis-example}\\
  \hline
\end{tabular}}
\end{table}

\section{seção 1}

