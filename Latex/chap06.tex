\chapter{Cronograma detalhado}

Em geral, o cronograma de execu\c{c}\~{a}o das atividades a serem feitas ao longo de um projeto
de pesquisa \'{e} apresentado na forma de tabela. \'{E} importante ressaltar que o projeto de pesquisa 
de mestrado \'{e} \'{e} avaliado no terceiro trimestre do primeiro ano. Nesta \'{e}poca, espera-se que
algumas atividades previstas no projeto j\'{a} tenham sido executadas.

A tabela abaixo usa alguns s\'{i}mbolos do pacote \verb|pifont| (\url{https://en.wikibooks.org/wiki/LaTeX/Special_Characters#Other_symbols}). O comando \verb|\ding{número}| insere um determinado s\'{i}mbolo.

\begin{table}[h]
\caption[]{Exemplo ilustrativo de um Cronograma de atividades a serem desenvolvidas durante o mestrado.
As siglas T1$-$T8 indicam os trimestres. A c\'{e}lulas em cinza representam o planejamento inicial. Os símbolos \ding{52} e \ding{45} indicam, respectivamente atividades conclu\'{i}das e em andamento. \\}
\label{tab:cronograma}
\centering
{\footnotesize
\begin{tabular}{|c|c|c|c|c|c|c|c|c|}
  \hline
  Atividade & T1 & T2 & T3 & T4 & T5 & T6 & T7 & T8 \\
  \hline
  Revis\~{a}o bibliogr\'{a}fica & \cellcolor{gray}\ding{52} & \cellcolor{gray}\ding{52} & & & & & & \\  
  Disciplinas & \cellcolor{gray}\ding{52} & \cellcolor{gray}\ding{52} & & & & & & \\
  Processamento dos dados 1 & & \cellcolor{gray}\ding{52} & \cellcolor{gray}\ding{45} & \cellcolor{gray} & & & & \\
  Processamento dos dados 2 & & \cellcolor{gray}\ding{52} & \cellcolor{gray}\ding{45} & \cellcolor{gray} & & & & \\
  Interpreta\c{c}\~{a}o dos resultados & & & & \cellcolor{gray} & \cellcolor{gray} & \cellcolor{gray} & \cellcolor{gray} & \\
  Escrita da disserta\c{c}\~{a}o & & & & & & & \cellcolor{gray} & \cellcolor{gray} \\
  \hline
\end{tabular}}
\end{table}

De acordo com o cronograma acima, as atividades ``Revis\~{a}o bibliogr\'{a}fica" e ``Disciplinas" j\'{a} foram conclu\'{i}das e as atividades ``Processamento dos dados 1" e ``Processamento dos dados 2" foram iniciadas no trimestre T2.
