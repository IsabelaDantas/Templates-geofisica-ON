\chapter{Introdução}

\textbf{OBS0: Testado com o MikTex 2.9 em Windows, rodando o PDFLaTex para geração diretamente do PDF da tese (anexado como exemplo). Como editor, indicamos o WinEdt. Entretanto, este template pode ser rodado em qualquer editor de texto e também no sistema operacional Linux.}

OBS1: Segundo a norma de formatação de teses e disserta{\c c}\~oes do
Programa de Pós-graduação em Geofísica do Observatório Nacional (ON/MCTI)
é obrigatório que toda abreviatura deva ser definida na primeira vez que é
utilizada, mas não é obrigatório colocar uma lista de abreviações no preâmbulo.
Entretanto, é altamente indicado que se coloque uma lista de abreviação no
preâmbulo, com todas as abreviações utilizadas no trabalho, uma vez que isto
torna o texto mais claro.

EXEMPLO.

Um exemplo de utilização de abreviação é dado na seguir: O Método das Diferenças Finitas
(MDF\abbrev{MDF}{Método dos Diferenças Finitas}) é um dos métodos numéricos mais eficientes
para a resolução de equações diferenciais...

Repare que, na primeira vez que a abreviação ocorre no texto, basta utilizar o comando
``\verb|\abbrev|'' para descrever a abreviação, sendo tal abreviação colocada automaticamente
e uma listagem de abreviações no preâmbulo do texto.

Do mesmo modo, pode-se definir os símbolos com o comando ``\verb|\symbl|'', tal como o
conjunto dos números reais $\mathbb{R}$ e o conjunto vazio $\emptyset$.
\symbl{$\mathbb{R}$}{Conjunto dos n\'umeros reais}
\symbl{$\emptyset$}{Conjunto vazio}

Após isto, antes de compilar o código com pdflatex por exemplo, é necessário rodar o make index
para gerar as listas, por exemplo, através dos seguinte comandos na linha de comando:

\verb|makeindex -s on.ist -o thesis.lab thesis.abx|

\verb|makeindex -s on.ist -o thesis.los thesis.syx|

